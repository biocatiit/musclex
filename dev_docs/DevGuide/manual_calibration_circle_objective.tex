\documentclass[11pt]{article}

\usepackage[T1]{fontenc}
\usepackage{lmodern}
\usepackage{amsmath,amssymb}

\title{Manual Calibration: Circle-band objective (contrast with background)}
\author{}
\date{}

\begin{document}
\maketitle

\noindent
This note documents the objective used by \texttt{musclex/ui/ManualCalibrationDialog.py} in
\texttt{ManualCalibrationDialog.\_circle\_band\_objective(center, radius, Q)} (the ``circle band'' optimizer).

\section{What it computes}

Given:
\begin{itemize}
  \item circle center \(c=(c_x,c_y)\)
  \item radius \(r\)
  \item band scale \(Q>0\) (roughly a ring-width scale, in pixels)
  \item \texttt{objective\_alpha} \(\alpha\)
  \item \texttt{objective\_bg\_k} \(k\)
  \item \texttt{objective\_nphi} \(N_\phi\) (number of angular samples)
\end{itemize}

Let angles be uniformly sampled:
\[
\phi_j = \frac{2\pi j}{N_\phi},\quad j=0,\ldots,N_\phi-1
\]

Let the bilinear-sampled image intensity at a point be \(I(x,y)\).

\subsection{Signal bands}

The ``signal'' is sampled on 5 concentric circles near the candidate ring, using offsets
\[
o \in \{-Q,\,-0.5Q,\,0,\,0.5Q,\,Q\}.
\]

The signal mean is
\[
\mu_{\text{sig}} =
\frac{1}{5N_\phi}
\sum_{o}\sum_{j}
I\!\left(c_x+(r+o)\cos\phi_j,\;c_y+(r+o)\sin\phi_j\right).
\]

\subsection{Background bands}

The ``background'' is sampled on 2 concentric circles farther away (symmetric inside/outside), using offsets
\[
o \in \{-kQ,\,+kQ\}.
\]

The background mean is
\[
\mu_{\text{bg}} =
\frac{1}{2N_\phi}
\sum_{o}\sum_{j}
I\!\left(c_x+(r+o)\cos\phi_j,\;c_y+(r+o)\sin\phi_j\right).
\]

\subsection{Objective value (maximize)}

The returned objective is:
\[
J(c,r;Q)=\mu_{\text{sig}}-\alpha\,\mu_{\text{bg}}.
\]

\section{Why subtract a symmetric background}

\subsection{The problem: ``maximize the integration'' is biased by radial background}

If you maximize a raw ring integral/mean, you are effectively optimizing
\[
\int \big(S(r,\phi) + B(r,\phi)\big)\,d\phi,
\]
where \(S\) is the ring/peak contribution you care about and \(B\) is baseline/background intensity.

In real diffraction images, \(B\) commonly varies strongly with radius (beam halo, small-angle scatter,
detector response, broad diffuse scatter). That means the optimizer can increase the integral by
moving the circle to a location where \(B\) is larger---even if the ring alignment is worse.

Concretely, small changes to center/radius can ``ride'' the background gradient and produce a larger
sum/mean without actually matching the ring. This can pull the solution toward bright halos,
broad low-\(q\) scatter, nearby rings with higher baseline, or detector shading artifacts.

\subsection{What the background bands do}

Subtracting \(\alpha\,\mu_{\text{bg}}\) makes the objective approximate \emph{local contrast}:
``how much brighter is the ring neighborhood compared to nearby off-ring samples''.

Sampling background at both \(r-kQ\) and \(r+kQ\) is important:
\begin{itemize}
  \item \textbf{Symmetry cancels first-order radial slope:} if \(B(r)\) is roughly smooth, then
  averaging inside/outside approximates the local baseline at \(r\) and reduces bias from \(dB/dr\).
  \item \textbf{Stays local but avoids the peak:} using \(\pm kQ\) ties the background separation to the ring-width
  scale \(Q\), so the off-ring samples remain ``nearby'' across different rings/images while being far
  enough not to sit on the peak itself.
\end{itemize}

\subsection{Why not subtract a full radial profile?}

A full radial background model can work, but it requires additional assumptions (masking peaks, robust
fits, handling anisotropy). The two-band symmetric sampling is a lightweight, optimization-friendly
proxy that is local (responds to local baseline), robust to global illumination changes, and cheap to compute.

\subsection{Assumptions and practical caveats}
The interpretation above relies on two practical assumptions:
\begin{itemize}
  \item \textbf{Background smoothness:} the baseline varies reasonably smoothly with radius so that
  sampling at \(r\pm kQ\) approximates the local baseline at \(r\).
  \item \textbf{Background samples are off-peak:} the offsets \(\pm kQ\) are far enough from the ring
  so they do not land on the ring itself or a nearby ring/feature.
\end{itemize}
If rings are thick (e.g., \(\sim\!20\) pixels wide) or nearby features exist, increase \(Q\) and/or \(k\)
so the background bands are truly outside the ring, or set \(\alpha=0\) to disable the background term.

\section{Parameter intuition}
\begin{itemize}
  \item \textbf{\(Q\):} sets the ``thickness'' scale of the signal sampling band; also scales how far away background is measured.
  \item \textbf{\(k\) (\texttt{objective\_bg\_k}):} how far from the ring to sample background (in units of \(Q\)).
  Larger \(k\) reduces contamination from the ring peak but can make the background less ``local''.
  \item \textbf{\(\alpha\) (\texttt{objective\_alpha}):} weight of the background penalty.
  Higher \(\alpha\) emphasizes contrast; lower \(\alpha\) behaves more like raw intensity maximization.
  \item \textbf{\(N_\phi\) (\texttt{objective\_nphi}):} angular sampling density; higher values reduce noise at higher compute cost.
\end{itemize}

\end{document}

