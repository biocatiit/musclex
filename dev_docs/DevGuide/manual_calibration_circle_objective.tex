\documentclass[11pt]{article}

\usepackage[T1]{fontenc}
\usepackage{lmodern}
\usepackage{amsmath,amssymb}

\title{Manual Calibration: Circle-band objective (contrast with background)\\
       and differential-evolution optimization}
\author{}
\date{}

\begin{document}
\maketitle

\noindent
This note documents the objective used by \texttt{musclex/ui/ManualCalibrationDialog.py} in
\texttt{ManualCalibrationDialog.\_circle\_band\_objective(center, radius, Q)} (the ``circle band'' optimizer).

\section{What it computes}

Given:
\begin{itemize}
  \item circle center \(c=(c_x,c_y)\)
  \item radius \(r\)
  \item band scale \(Q>0\) (roughly a ring-width scale, in pixels)
  \item \texttt{objective\_alpha} \(\alpha\)
  \item \texttt{objective\_bg\_k} \(k\)
  \item \texttt{objective\_nphi} \(N_\phi\) (number of angular samples)
\end{itemize}

Let angles be uniformly sampled:
\[
\phi_j = \frac{2\pi j}{N_\phi},\quad j=0,\ldots,N_\phi-1
\]

Let the bilinear-sampled image intensity at a point be \(I(x,y)\).

\subsection{Signal bands}

The ``signal'' is sampled on 5 concentric circles near the candidate ring, using offsets
\[
o \in \{-Q,\,-0.5Q,\,0,\,0.5Q,\,Q\}.
\]

\subsection{Background bands}

The ``background'' is sampled on 2 concentric circles farther away (symmetric inside/outside), using offsets
\[
o \in \{-kQ,\,+kQ\}.
\]

\subsection{Per-angle averaging}

For each angular sample \(\phi_j\) the signal and background intensities are averaged
over the respective radial offsets:
\[
s_j = \frac{1}{5}\sum_{o\,\in\,\text{sig}} I\!\left(c_x+(r+o)\cos\phi_j,\;c_y+(r+o)\sin\phi_j\right),
\qquad
b_j = \frac{1}{2}\sum_{o\,\in\,\text{bg}} I\!\left(c_x+(r+o)\cos\phi_j,\;c_y+(r+o)\sin\phi_j\right).
\]

\subsection{MAD-based outlier rejection}

Before computing the final means, outlier angles are rejected using
Median Absolute Deviation (MAD) filtering on the per-angle signal values
\(\{s_j\}\):
\begin{enumerate}
  \item Compute the median: \(m = \operatorname{median}(s_j)\).
  \item Compute the MAD: \(\mathrm{MAD} = \operatorname{median}\!\bigl(|s_j - m|\bigr)\).
  \item Define the ``good'' set
    \(\mathcal{G} = \bigl\{j : |s_j - m| \le 3\,\mathrm{MAD}\bigr\}\).
    (If \(\mathrm{MAD} < 10^{-10}\), all angles are kept.)
  \item Apply the same angular mask to both signal and background for consistency.
\end{enumerate}

\subsection{Objective value (maximize)}

The returned objective is computed over the surviving angles:
\[
J(c,r;Q)
= \frac{1}{|\mathcal{G}|}\sum_{j\in\mathcal{G}} s_j
  \;-\;\alpha\,\frac{1}{|\mathcal{G}|}\sum_{j\in\mathcal{G}} b_j.
\]

\section{Why subtract a symmetric background}

\subsection{The problem: ``maximize the integration'' is biased by radial background}

If you maximize a raw ring integral/mean, you are effectively optimizing
\[
\int \big(S(r,\phi) + B(r,\phi)\big)\,d\phi,
\]
where \(S\) is the ring/peak contribution you care about and \(B\) is baseline/background intensity.

In real diffraction images, \(B\) commonly varies strongly with radius (beam halo, small-angle scatter,
detector response, broad diffuse scatter). That means the optimizer can increase the integral by
moving the circle to a location where \(B\) is larger---even if the ring alignment is worse.

Concretely, small changes to center/radius can ``ride'' the background gradient and produce a larger
sum/mean without actually matching the ring. This can pull the solution toward bright halos,
broad low-\(q\) scatter, nearby rings with higher baseline, or detector shading artifacts.

\subsection{What the background bands do}

Subtracting \(\alpha\,\mu_{\text{bg}}\) makes the objective approximate \emph{local contrast}:
``how much brighter is the ring neighborhood compared to nearby off-ring samples''.

Sampling background at both \(r-kQ\) and \(r+kQ\) is important:
\begin{itemize}
  \item \textbf{Symmetry cancels first-order radial slope:} if \(B(r)\) is roughly smooth, then
  averaging inside/outside approximates the local baseline at \(r\) and reduces bias from \(dB/dr\).
  \item \textbf{Stays local but avoids the peak:} using \(\pm kQ\) ties the background separation to the ring-width
  scale \(Q\), so the off-ring samples remain ``nearby'' across different rings/images while being far
  enough not to sit on the peak itself.
\end{itemize}

\subsection{Why MAD outlier rejection?}

Real diffraction images often contain beam stops, detector gaps, parasitic scatter, or
saturated pixels that produce extreme intensity values at particular angles.
Without filtering, a handful of such angles can dominate the mean and steer the optimizer
toward solutions that maximize these artifacts rather than the ring contrast.

MAD filtering is chosen because:
\begin{itemize}
  \item It is \emph{robust}: the median and MAD are not affected by a minority of extreme
    values, unlike the standard deviation.
  \item It is \emph{cheap}: only two passes over the per-angle array are needed (median, then
    median of absolute deviations).
  \item The threshold \(3\,\mathrm{MAD}\) is a conventional choice for moderate outlier
    rejection without discarding too many valid samples.
\end{itemize}

The same angular mask is applied to both signal and background arrays so that
the contrast estimate remains paired (each surviving angle contributes to both
the numerator and the background penalty).

\subsection{Why not subtract a full radial profile?}

A full radial background model can work, but it requires additional assumptions (masking peaks, robust
fits, handling anisotropy). The two-band symmetric sampling is a lightweight, optimization-friendly
proxy that is local (responds to local baseline), robust to global illumination changes, and cheap to compute.

\subsection{Assumptions and practical caveats}
The interpretation above relies on two practical assumptions:
\begin{itemize}
  \item \textbf{Background smoothness:} the baseline varies reasonably smoothly with radius so that
  sampling at \(r\pm kQ\) approximates the local baseline at \(r\).
  \item \textbf{Background samples are off-peak:} the offsets \(\pm kQ\) are far enough from the ring
  so they do not land on the ring itself or a nearby ring/feature.
\end{itemize}
If rings are thick (e.g., \(\sim\!20\) pixels wide) or nearby features exist, increase \(Q\) and/or \(k\)
so the background bands are truly outside the ring, or set \(\alpha=0\) to disable the background term.

\section{Optimization strategy: differential evolution}

The optimizer in
\texttt{ManualCalibrationDialog.\_optimize\_center\_radius(center, radius, Q)}
refines the initial circle fit (from user-selected points) by maximizing the
circle-band objective \(J(c,r;Q)\) over three parameters:
center \((c_x, c_y)\) and radius \(r\).

\subsection{Search bounds}

The search region is defined relative to the initial fit:
\[
c_x \in [\hat{c}_x - \Delta c,\;\hat{c}_x + \Delta c],\qquad
c_y \in [\hat{c}_y - \Delta c,\;\hat{c}_y + \Delta c],\qquad
r   \in [\hat{r}(1-\epsilon),\;\hat{r}(1+\epsilon)],
\]
where \(\Delta c = 100\) pixels and \(\epsilon = 0.10\) (10\%).

\subsection{Solver}

The optimization uses \texttt{scipy.optimize.differential\_evolution} with the
following settings:
\begin{itemize}
  \item \textbf{Population size:} 100 (per parameter dimension).
  \item \textbf{Tolerance:} \(10^{-4}\) (convergence criterion on objective
    improvement).
  \item \textbf{Polish:} enabled---after convergence, a local L-BFGS-B refinement
    is applied to the best solution.
  \item \textbf{Updating:} \texttt{deferred} (full-generation updates; better for
    smooth, non-noisy landscapes).
  \item \textbf{Seed:} 42 (reproducible results).
\end{itemize}

Because the objective is negated for the minimizer, the solver effectively
\emph{maximizes} \(J\).

\subsection{Why differential evolution?}

The previous implementation used an alternating coordinate-descent approach
(refine center \(\to\) refine radius \(\to\) repeat) with a grid-search over
8-connected neighbors and shrinking step sizes.

Differential evolution replaces this with a \emph{global} optimizer that:
\begin{itemize}
  \item Explores the full 3-D search region simultaneously, avoiding local
    optima that depend on the initial point quality.
  \item Does not require manual tuning of step sizes, shrink factors, or
    alternation schedules.
  \item Terminates with a local polish step that achieves sub-pixel precision.
\end{itemize}

The trade-off is higher computational cost per call (hundreds to thousands of
objective evaluations), but for a one-shot calibration step this is
acceptable---typical runs complete in under a second on modern hardware.

\section{Parameter intuition}
\begin{itemize}
  \item \textbf{\(Q\):} sets the ``thickness'' scale of the signal sampling band; also scales how far away background is measured.
  \item \textbf{\(k\) (\texttt{objective\_bg\_k}):} how far from the ring to sample background (in units of \(Q\)).
  Larger \(k\) reduces contamination from the ring peak but can make the background less ``local''.
  \item \textbf{\(\alpha\) (\texttt{objective\_alpha}):} weight of the background penalty.
  Higher \(\alpha\) emphasizes contrast; lower \(\alpha\) behaves more like raw intensity maximization.
  \item \textbf{\(N_\phi\) (\texttt{objective\_nphi}):} angular sampling density; higher values reduce noise at higher compute cost.
\end{itemize}

\end{document}

